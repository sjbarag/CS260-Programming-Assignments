\documentclass{article}

\usepackage[margin=1in]{geometry}
\usepackage{fancyhdr}
\usepackage{float}
\usepackage{multirow}
\usepackage{rotating}
\usepackage{siunitx}
\usepackage{amsmath}

\pagestyle{fancy}
\lhead{CS 260}
\chead{Programming Assignment 2: Hash Table Probe Analysis}
\rhead{Sean Barag}

\title{Programming Assignment 2: Hash Table Probe Analysis}
\author{Sean Barag\\ \texttt{<sjb89@drexel.edu>}}

\newcommand{\tbf}[1]{\textbf{#1}}

\begin{document}
\maketitle
Programming assignment two required students to compare the open and closed
hash tables as a means of storing a dictionary.  This was measured in the
average number of probes required to insert and delete fixed number of elements
to a hash table with varying number of buckets.

\section{Open Hashing}
The complete Project Gutenberg Etext of \emph{Alice's Adventures In Wonderland}
was used as an input to the hash table.  With the number of buckets ranging
from just one up to 15,000 (an arbitrary point at which the average number of
probes appeared to stabilize) at increments of 250, each word in the input was
inserted into the table.  Once all words had been successfully added, the input
was again parsed so that each word could be deleted.  The results of this
experiment can be seen in Table~\ref{tab:openData}.
%

\begin{table}[H]
	\footnotesize
	\centering
	\begin{tabular}{|c|c|c|c|c|}
		\hline
				& \tbf{Insert} & \tbf{Delete} & \tbf{Insert} & \tbf{Delete} \\
				& \tbf{Total}  & \tbf{Total}  & \tbf{Average}& \tbf{Average}\\
		\tbf{B} & \tbf{Probes} & \tbf{Probes} & \tbf{Probes} &\tbf{Probes}  \\ \hline
		1       & 85182903     & 89749809     & 2822.86926697         &  2486.97098759 \\ \hline
		250     & 370874       & 354549       & 12.2903632025         &  10.3333916237 \\ \hline
		500     & 197411       & 178054       & 6.54198700954         &  5.57394189832 \\ \hline
		750     & 132108       & 116625       & 4.37791622481         &  3.70050133266 \\ \hline
		1000    & 108861       & 84298        & 3.60753579003         &  3.00516915618 \\ \hline
		1250    & 100875       & 74059        & 3.34288838812         &  2.53314406896 \\ \hline
		1500    & 78721        & 58611        & 2.60872879109         &  2.25496306556 \\ \hline
		1750    & 74531        & 51440        & 2.46987672322         &  2.00272532607 \\ \hline
		2000    & 72315        & 46447        & 2.39644088017         &  1.9908701243  \\ \hline
		2250    & 59252        & 36705        & 1.96354718982         &  1.80910838385 \\ \hline
		2500    & 64694        & 35551        & 2.14388918346         &  1.70173759035 \\ \hline
		2750    & 59473        & 32954        & 1.97087089077         &  1.62575234336 \\ \hline
		3000    & 54585        & 31027        & 1.8088878579          &  1.52152805022 \\ \hline
		3250    & 57631        & 30673        & 1.90982900318         &  1.56999539336 \\ \hline
		3500    & 51873        & 27345        & 1.71901511135         &  1.43731931669 \\ \hline
		3750    & 49325        & 27350        & 1.6345771474          &  1.44525470302 \\ \hline
		4000    & 52278        & 25636        & 1.73243637328         &  1.4238267148  \\ \hline
		4250    & 46724        & 22667        & 1.54838282078         &  1.35560074158 \\ \hline
		4500    & 45270        & 22177        & 1.50019883351         &  1.34863780102 \\ \hline
		4750    & 47160        & 23324        & 1.56283138918         &  1.33968983343 \\ \hline
		5000    & 46325        & 18900        & 1.53516039236         &  1.27944760357 \\ \hline
		5250    & 44251        & 20304        & 1.46643027572         &  1.21203438395 \\ \hline
		5500    & 43998        & 20132        & 1.45804612937         &  1.27224469161 \\ \hline
		5750    & 41806        & 17413        & 1.38540562036         &  1.21735178971 \\ \hline
		6000    & 43213        & 18502        & 1.43203207847         &  1.24399919317 \\ \hline
		6250    & 43030        & 19479        & 1.42596765642         &  1.25719633406 \\ \hline
		6500    & 41869        & 17020        & 1.38749337222         &  1.17850713198 \\ \hline
		6750    & 39501        & 15217        & 1.30902041357         &  1.19988960732 \\ \hline
		7000    & 39729        & 15391        & 1.31657608696         &  1.16793140082 \\ \hline
		7250    & 47020        & 15940        & 1.55819194062         &  1.14453938393 \\ \hline
		7500    & 40456        & 14795        & 1.34066808059         &  1.17401999683 \\ \hline
		7750    & 40029        & 16200        & 1.32651776246         &  1.18093016475 \\ \hline
		8000    & 42967        & 17215        & 1.42387990456         &  1.21764040175 \\ \hline
		8250    & 39874        & 14917        & 1.32138123012         &  1.12589629406 \\ \hline
		8500    & 39589        & 14626        & 1.31193663839         &  1.11717079132 \\ \hline
		8750    & 38097        & 12398        & 1.26249337222         &  1.11623300621 \\ \hline
		9000    & 37258        & 13182        & 1.23468981972         &  1.12158597805 \\ \hline
		9250    & 41438        & 12762        & 1.37321049841         &  1.12065331928 \\ \hline
		9500    & 39011        & 14576        & 1.29278234358         &  1.21103356597 \\ \hline
		9750    & 38682        & 14760        & 1.28187963945         &  1.14135477884 \\ \hline
		10000   & 39054        & 12412        & 1.29420731707         &  1.11119068935 \\ \hline
		10250   & 36220        & 11764        & 1.20029162248         &  1.12628051699 \\ \hline
		10500   & 36358        & 11551        & 1.20486479321         &  1.09789943922 \\ \hline
		10750   & 39010        & 14729        & 1.29274920467         &  1.12935132648 \\ \hline
		11000   & 35758        & 11914        & 1.18498144221         &  1.08555808656 \\ \hline
		11250   & 36326        & 10789        & 1.20380434783         &  1.0965545279  \\ \hline
		11500   & 36250        & 12026        & 1.20128579003         &  1.07260078487 \\ \hline
		11750   & 37949        & 12937        & 1.2575888123          &  1.08677755376 \\ \hline
		12000   & 36712        & 12605        & 1.21659597031         &  1.07901044342 \\ \hline
		12250   & 35662        & 11924        & 1.18180010604         &  1.05859375    \\ \hline
		12500   & 36486        & 10834        & 1.20910657476         &  1.07246089883 \\ \hline
		12750   & 35065        & 11154        & 1.16201617179         &  1.0765370138  \\ \hline
		13000   & 35487        & 10318        & 1.17600079533         &  1.07289175419 \\ \hline
		13250   & 37766        & 9736         & 1.25152439024         &  1.05219928672 \\ \hline
		13500   & 35063        & 10956        & 1.16194989396         &  1.07972799842 \\ \hline
		13750   & 37340        & 11525        & 1.23740721103         &  1.06604384423 \\ \hline
		14000   & 34626        & 10414        & 1.14746818664         &  1.06657107743 \\ \hline
		14250   & 35173        & 12312        & 1.16559517497         &  1.04961636829 \\ \hline
		14500   & 38943        & 11923        & 1.29052889714         &  1.06894387664 \\ \hline
		14750   & 34147        & 9975         & 1.13159464475         &  1.07431340872 \\ \hline
		15000   & 35340        & 10186        & 1.17112937434         &  1.05162089614 \\ \hline
	\end{tabular}
	\parbox{.55\textwidth}{\caption{Probe data resulting from the open hashing of \emph{Alice's Adventures in Wonderland}.}
	\label{tab:openData}}
\end{table}

%
The expected trend for both insertion and deletion for this test is $O(1 +
\frac{N}{B})$, where $N$ is the number of elements being inserted (in this
case, 28,198 words according to Bash \texttt{wc -l}).  While the data certainly
decays, it does so at a much faster rate than is expected, resembling a
logarithmic function more closely than the provided formula.  This is most
likely due to the underlying data structure of a linked list, which requires
$O(\log_2 n)$ time to traverse.  Since each bucket potentially contains a fully
qualified linked list and a probe was considered one comparison of a linked
list node to the word in question, the measured logarithmic behavior is not
very surprising.

\section{Closed Hashing}
Closed hashing was tested in a nearly identical way to open hashing:
\emph{Alice in Wonderland} was parsed, with each word getting added to the
closed hash table where space allowed.  At the end of the insertions, the story
was again parsed so that each word could be deleted.  $B$, the number of
buckets, was varied from one to 34,000 in increments of 1,000. The resulting
total and average number of probes is shown in Table~\ref{tab:closedData}.
%
\begin{table}[H]
	\footnotesize
	\centering
	\begin{tabular}{|c|c|c|c|c|}
		\hline
				& \tbf{Insert} & \tbf{Delete} & \tbf{Insert} & \tbf{Delete} \\
				& \tbf{Total}  & \tbf{Total}  & \tbf{Average}& \tbf{Average}\\
		\tbf{B} & \tbf{Probes} & \tbf{Probes} & \tbf{Probes} &\tbf{Probes}  \\ \hline
		1       & 30174        & 0            &  0.872534844717         & 0.0 \\ \hline
		1000    & 12767845     & 12626        &  369.204933202          & 0.365103232896   \\ \hline
		2000    & 14261813     & 7899         &  412.405673472          & 0.228413625586   \\ \hline
		3000    & 14034777     & 5249         &  405.840523972          & 0.151784165173   \\ \hline
		4000    & 11295085     & 4778         &  326.61745995           & 0.138164362963   \\ \hline
		5000    & 6209139      & 3768         &  179.548291018          & 0.108958417674   \\ \hline
		6000    & 241348       & 2298         &  6.97900641952          & 0.0664507547279  \\ \hline
		7000    & 32741        & 1384         &  0.946764212596         & 0.0400208200798  \\ \hline
		8000    & 19917        & 1186         &  0.575935457753         & 0.034295298132   \\ \hline
		9000    & 10662        & 780          &  0.308310681858         & 0.0225550864612  \\ \hline
		10000   & 11142        & 791          &  0.322190735064         & 0.0228731710138  \\ \hline
		11000   & 6770         & 556          &  0.195766583772         & 0.016077728298   \\ \hline
		12000   & 5295         & 457          &  0.153114336938         & 0.013214967324   \\ \hline
		13000   & 5377         & 435          &  0.155485512694         & 0.0125787982187  \\ \hline
		14000   & 4342         & 268          &  0.125556647967         & 0.00774969637384 \\ \hline
		15000   & 4102         & 406          &  0.118616621364         & 0.0117402116708  \\ \hline
		16000   & 3662         & 281          &  0.105893239257         & 0.00812561448152 \\ \hline
		17000   & 3120         & 320          &  0.0902203458447        & 0.00925336880458 \\ \hline
		18000   & 2862         & 196          &  0.082759817246         & 0.00566768839281 \\ \hline
		19000   & 2565         & 178          &  0.0741715343242        & 0.00514718639755 \\ \hline
		20000   & 2828         & 272          &  0.0817766468105        & 0.00786536348389 \\ \hline
		21000   & 2586         & 177          &  0.074778786652         & 0.00511826962003 \\ \hline
		22000   & 2168         & 137          &  0.062691573651         & 0.00396159851946 \\ \hline
		23000   & 2518         & 170          &  0.072812445781         & 0.00491585217743 \\ \hline
		24000   & 1972         & 111          &  0.0570238852582        & 0.00320976230409 \\ \hline
		25000   & 2294         & 252          &  0.0663350876178        & 0.00728702793361 \\ \hline
		26000   & 1768         & 101          &  0.0511248626453        & 0.00292059452895 \\ \hline
		27000   & 1576         & 88           &  0.0455728413626        & 0.00254467642126 \\ \hline
		28000   & 1822         & 62           &  0.0526863686311        & 0.00179284020589 \\ \hline
		29000   & 1548         & 62           &  0.0447631715922        & 0.00179284020589 \\ \hline
		30000   & 1577         & 81           &  0.0456017581401        & 0.00234225897866 \\ \hline
		31000   & 1308         & 37           &  0.0378231449887        & 0.00106992076803 \\ \hline
		32000   & 1619         & 65           &  0.0468162627957        & 0.00187959053843 \\ \hline
		33000   & 1316         & 71           &  0.0380544792088        & 0.00205309120352 \\ \hline
		34000   & 1272         & 50           &  0.0367821409982        & 0.00144583887572 \\ \hline
	\end{tabular}
	\parbox{.55\textwidth}{\caption{Probe data resulting from the closed hashing of \emph{Alice's Adventures in Wonderland}.}
	\label{tab:closedData}}
\end{table}

%
Insertion for closed hash tables is expected to require approximately
$\frac{1}{2} \cdot \left(1 + \frac{1}{1-\frac{N}{B}}\right)$ probes, whereas
deletion should need roughly $\frac{1}{2} \cdot \left(1 +
\frac{1}{\left(1-\frac{N}{B}\right)^2}\right)$ probes of the table.  It is
clear in cases where $B < N$ that this should result in a negative number of
probes, something that is quite obviously not possible and which did not occur.
This does not occur for deletions as a result of the polynomial in the
denominator.  Of note in particular is the fact that zero probes were required
for a bucket of size one, as only one word can be stored in such a dictionary.

\section{Conclusions}
In conclusion, open hashing has a much more predictable behavior than closed
hashing, at least on the scope of this trial.  The drastic error in results
shown here is most likely due to the low (relative to the total word count)
number of buckets used throughout the experiment.
\end{document}
